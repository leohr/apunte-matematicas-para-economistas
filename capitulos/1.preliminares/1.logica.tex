
\section{Lógica}

La lógica le proporciona a las matemáticas un lenguaje claro y un método preciso para demostrar nuevas afirmaciones a partir de otras ya establecidas o dadas. A las afirmaciones previamente aceptadas las denominamos \textbf{Axiomas}. A aquellas que se construyen mediante la lógica las denominamos \textbf{Teoremas}. 

\subsection{Conectores lógicos}

\begin{definicion}
	\textbf{(Proposición lógica)}
	Una \textbf{proposición lógica} es una afirmación que siempre toma uno de los dos valores de verdad posibles: verdadero (V) o falso (F). 
\end{definicion}

\begin{ejemplo}
	Son proposiciones lógicas: 
	\begin{itemize}
		\item $2+1 = 5$. 
		\item $1 \geq 0$.
		\item Hoy está lloviendo. 
		\item Todo juego en forma normal admite al menos un Equilibrio de Nash.   
		\item Un monopolio perfectamente discriminador extrae todo el excedente del consumidor. 
	\end{itemize}
\end{ejemplo}

Los conectivos lógicos permiten crear nuevas proposiciones a partir de otras ya conocidas. El valor de verdad de la nueva proposición dependerá de los valores de verdad de sus componentes. 

\begin{definicion}
	\textbf{(Negación)} 
	Sea $p$ una proposición lógica. La proposición $\overline{p}$ (también denotada $\sim p$) se lee ``no $p$'' y es aquella cuyo valor de verdad siempre es distinto al de $p$. 
\end{definicion}

\begin{definicion}
	\textbf{(O lógico)}	
	Sean $p$ y $q$ proposiciones lógicas. La proposición $p \vee q$ se lee ``$p$ o $q$'' y es verdadera cuando al menos una de las proposiciones $p$ o $q$ es verdadera. 
\end{definicion}

\begin{definicion}
	\textbf{(Y lógico)} 
	Sean $p$ y $q$ proposiciones lógicas. La proposición $p \wedge q$ se lee ``$p$ y $q$'' y es verdadera cuando tanto $p$ y $q$ son verdaderas. 
\end{definicion}

\begin{definicion}
	\textbf{(Implicancia)}
	Sean $p$ y $q$ proposiciones lógicas. La proposición $p \Rightarrow q$ se lee ``$p$ implica $q$'' o ``si $p$, entonces $q$''. Es falsa solo cuando $p$ es verdadera y $q$ es falsa. En cualquier otro caso, es verdadera. 
	
	A $p$ se le llama la \textbf{hipótesis} y a $q$ la \textbf{conclusión} de la proposición $p\Rightarrow q$. 
\end{definicion}

\begin{definicion}
	\textbf{(Equivalencia)}
	Sean $p$ y $q$ proposiciones lógicas. La proposición $p\Leftrightarrow q$ se lee ``$p$ es equivalente a $q$'' o ``$p$ si y solo si $q$''. Es verdadera cuando $p$ y $q$ tienen el mismo valor de verdad y falsa cuando estos valores difieren. 
\end{definicion}

\begin{ejemplo}
	Algunos ejemplos de las definiciones anteriores son: 
	\begin{itemize}
		\item La negación de la proposición ``estoy poniendo atención'' es ``no estoy poniendo atención''. 
		\item La proposición lógica ``mañana lloverá o mañana no lloverá'' es verdadera.  
		\item La proposición lógica ``mañana lloverá y mañana no lloverá'' es falsa. 
		\item La proposición lógica ``si estoy en Santiago, entonces estoy en Chile'' es verdadera.
		\item Si $a$ y $b$ son números reales, la proposición $a \geq b \Longleftrightarrow 2a \geq 2b$ es verdadera. 
	\end{itemize}
\end{ejemplo}

\begin{definicion}
	\textbf{(Tautología)}
	Una \textbf{tautología} es una proposición que, sin importar el valor de verdad de las proposiciones que la constituyen, es siempre verdadera. 
	
	Un argumento que permite saber que una proposición es una tautología se llama una \textbf{demostración}. 
\end{definicion}

\begin{proposicion}
	Son tautologías: 
	\begin{itemize}
		\item $p \o V \iff V , \quad p \y F \iff F$. 
		\item $p \y  V \iff p, \quad p \vee F \iff p$. 
		\item $p \y p \iff p, \quad p \o p \iff p$. 
		\item $\sim (\sim p) \iff p$. 
		\item $p \o \sim p \iff V$. 
		\item $p \y \sim p \iff F$. 
		\item $(p \Rightarrow q) \iff \sim p \o q$. 
	\end{itemize}
\end{proposicion}

\begin{proposicion}
	Las siguientes son tautologías: 
	\begin{itemize}
		\item Leyes de Morgan: $$\overline{p\y q} \iff \overline{p}\o \overline{q}, \quad \overline{p\o q} \iff \overline{p}\y \overline{q}$$
		\item Conmutatividad: $$p \o q \iff q \o p , \quad p \y q \iff q \y p$$
		\item Transitividad: $$[(p \Rightarrow q) \y (q \Rightarrow r
		)] \Rightarrow (p \Rightarrow r )$$
		$$[(p \iff q) \y (q \iff r)] \Rightarrow (p \iff r )$$
		\item Equivalencia dividida: $$ ( p \iff q) \iff ( p \Rightarrow q ) \y ( q \Rightarrow p)$$ 
		\item Demostración por casos: 
		$$ (p \o q \Rightarrow r) \iff ( p \Rightarrow r) \y ( q \Rightarrow r) $$ 
		\item Demostración por contradicción: 
		$$ (p \Rightarrow q ) \iff ( p \y \sim q \Rightarrow F ) $$
		\item Contrarrecíproca: 
		$$ (p \Rightarrow q) \iff ( \sim q \Rightarrow \sim p ) $$ 
	\end{itemize}
\end{proposicion}

\begin{ejemplo}
	Sabemos que $a$ es un número racional si puede escribirse como $a = m/n$ con $m$ y $n$ números enteros y $n \neq 0$. 
	
	Veamos que: 
	$$ 3a \text{ no es racional} \Rightarrow a \text{ no es racional} $$
	Por contrarrecíproca, sabemos que lo anterior es equivalente a probar que: 
	$$ a \text{ es racional } \Rightarrow 3a \text{ es racional } $$ 
	Veamos que esto último es cierto: 
	\begin{itemize}
		\item Supongamos que $a$ es racional (recordar que $p \Rightarrow q$ es falso solo cuando $p$ es verdadero y $q$ es falso).
		\item Luego, $a = m/n$ con $m$ y $n$ enteros con $n\neq 0$.
		\item Por lo tanto, $3 a = 3m /n$.
		\item Y como $3a = m'/n$ con $m'$ y $n$ enteros con $n \neq 0$, se concluye que $3a$ es un número racional. $\qed$ 
	\end{itemize}
\end{ejemplo}

\begin{ejemplo}
	Mostraremos que si el producto de dos números $a$ y $b$ es $1/3$, entonces ambos son racionales o ninguno lo es. 
	
	Razonando por contradicción, supongamos que: 
	$$ a \cdot b = 1 /3 $$ 
	Y que $a$ es racional, pero $b$ no lo es (notar que esto es \textbf{sin pérdida de generalidad}).
	
	Como $a$ es racional, entonces existen $m$, $n$ números enteros con $n \neq 0$ tales que: 
	$$ a = m / n $$ 
	
	Entonces: 
	$$ \frac{m}{n} \cdot b = \frac{1}{3} \quad \Rightarrow \quad b = \frac{n}{3m} $$ 
	Pero esto significa que $b$ es un número racional ($n$, $3m$ enteros con $3m\neq 0$), lo que es una contradicción y con lo que se concluye lo pedido. $\qed$ 
	
	¿Porqué $3 m \neq 0$? 
\end{ejemplo}

\subsection{Cuantificadores proposicionales}

La lógica proposicional que hemos visto nos permite hacer deducciones y construir propiedades. Más aún, nos permite deducir que si $p_1 \y p_2 \y ... \y p_n$ es una proposición verdadera, entonces debe ocurrir que $p_1, p_2, ... , p_n$ son proposiciones verdaderas. Notar que la afirmación anterior es una tautología. ¿Cuál es la demostración? Sin embargo, no tenemos el lenguaje matemático para explicitar que todas las proposiciones $p_i$ son verdaderas. 

\begin{definicion}
	Una \textbf{función proposicional (o predicado)}  $P(x, y, z, ...)$ es una expresión que depende de cero o más variables $x, y, z, ...$ que al ser reemplazadas por elementos de un conjunto de referencia $E$ hacen que $P$ se transforme en una proposición lógica. Es decir, que sea verdadera o falsa. 
\end{definicion}

\begin{ejemplo}
	Algunos ejemplos de funciones proposicionales son: 
	\begin{itemize}
		\item $P(x):$ ``$x$ es futbolista'' es un predicado sobre el conjunto de referencia de personas. Notar que $P(\text{Christiane Endler})$ es verdadera y que $P(\text{Nicolás Massu})$ es falsa. 
		\item $P(a,b,c): 2a + b \geq c$ es un predicado sobre el conjunto de números reales.
	\end{itemize}
\end{ejemplo}

\begin{definicion}
	\textbf{(Cuantificador universal)}
	La expresión $\forall x \in E, P(x)$, que se lee ``para todo $x$ en $E$, $p(x)$'' es una proposición verdadera si al reemplazar $x$ por cualquier elemento de $E$ se verifica que $P(x)$ es verdadera. 
\end{definicion}

\begin{definicion}
	\textbf{(Cuantificador existencial)}
	La expresión $\exists x \in E, P(x)$, que se lee ``existe $x$ en $E$ tal que $P(x)$'', es una proposición que es verdadera si y solo si $\forall x \in E , \sim P(x)$ es falsa. 
	
	En otras palabras, $\exists x \in E, P(x)$ será verdadera si es que algún elemento $e$ de $E$ hace que $P(e)$ sea verdadera. 
\end{definicion}

\begin{definicion}
	\textbf{(Existencia y Unicidad)}
	La expresión $\exists ! x \in E, P(x)$ que se lee ``existe un único $x$ en $E$ tal que $P(x)$'', es una proposición que es verdadera si y solo si existe un $e$ en $E$ tal que $P(e)$ es verdadera y para todo $x \neq e$, $P(x)$ es falsa. 
\end{definicion}

\begin{proposicion}
	Sea $P(x,y)$ un predicado con conjunto de referencia $E$. Se cumplen las siguientes tautologías: 
	\begin{itemize}
		\item $\forall x \in E, \forall y \in E, P(x,y) \iff \forall y \in E, \forall x \in E, P(x,y)$. 
		\item $\exists x \in E, \exists y \in E, P(x,y) \iff \exists y \in E, \exists x \in E, P(x,y)$
		\item $\exists x \in E, \forall y \in E, P(x,y) \Longrightarrow \forall y \in E, \exists x \in E, P(x,y)$. 
	\end{itemize}
\end{proposicion}

\begin{nota}
	Notar que: 
	\begin{itemize}
		\item Las afirmaciones de la proposición anterior son válidas para cualquier predicado $P$! 
		\item La recíproca de la última implicancia no es cierta: basta tomar $E = \N$ y $P(x,y): x > y$ para tener el contraejemplo. 
	\end{itemize}
\end{nota}

\subsection{Referencias}

\begin{itemize}	
	\item \textcolor{red}{Apunte Introducción al Álgebra, semanas 1 y 2. }
\end{itemize}