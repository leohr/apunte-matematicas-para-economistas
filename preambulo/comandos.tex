
% Ambientes
\newtheorem{teorema}{Teorema}[chapter]

\newtheorem{definicion}{Definición}[chapter]
\newtheorem{proposicion}{Proposición}[chapter]
\newtheorem{lema}{Lema}[chapter]
\newtheorem{corolario}{Corolario}[chapter] 

\newtheorem{observacion}{Observacion}[chapter] 
\newtheorem{nota}{Nota}[chapter] 

\theoremstyle{remark}
\newtheorem{ejemplo}{Ejemplo}[chapter]

\newcounter{problema}[section]
\newenvironment{problema}[2][]{\refstepcounter{problema}\par\medskip
	\noindent \textbf{Problema~\theproblema. #1 #2} \rmfamily}{\medskip}


% Comandos 
\newcommand{\N}{\mathbb{N}}
\newcommand{\R}{\mathbb{R}}
\newcommand{\Z}{\mathbb{Z}}
\newcommand{\Q}{\mathbb{Q}}

\renewcommand{\P}{\mathbb{P}}
\newcommand{\E}{\mathbb{E}}

\DeclareMathOperator*{\argmax}{argmax} 
\DeclareMathOperator*{\argmin}{argmin} 

% Operadores lógicos 
\newcommand{\y}{\wedge}
\renewcommand{\o}{\vee}